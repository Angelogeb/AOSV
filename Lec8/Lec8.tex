\documentclass[twoside]{article}
\setlength{\oddsidemargin}{-0.5 in}
\setlength{\evensidemargin}{1.5 in}
\setlength{\topmargin}{-0.6 in}
\setlength{\textwidth}{5.5 in}
\setlength{\textheight}{8.5 in}
\setlength{\headsep}{0.5 in}
\setlength{\parindent}{0 in}
\setlength{\parskip}{0.07 in}
\setlength{\marginparwidth}{145pt}

%
% ADD PACKAGES here:
% 12

\usepackage{amsmath,
            amsfonts,
            amssymb,
            graphicx,
            mathtools,
            flexisym,
            marginnote,
            hyperref,
            titlesec}

\usepackage[english]{babel}
\usepackage[utf8]{inputenc}
\usepackage[shortlabels]{enumitem}

\graphicspath{ {images/} }

\hypersetup{
    colorlinks=true,
    linkcolor=blue,
    filecolor=magenta,      
    urlcolor=blue,
}

\titlespacing\section{0pt}{12pt plus 4pt minus 2pt}{0pt plus 2pt minus 2pt}
\titlespacing\subsection{0pt}{12pt plus 4pt minus 2pt}{0pt plus 2pt minus 2pt}

%
% The following commands set up the lecnum (lecture number)
% counter and make various numbering schemes work relative
% to the lecture number.
%
\newcounter{lecnum}
\renewcommand{\thepage}{\thelecnum-\arabic{page}}
\renewcommand{\thesection}{\thelecnum.\arabic{section}}
\renewcommand{\theequation}{\thelecnum.\arabic{equation}}
\renewcommand{\thefigure}{\thelecnum.\arabic{figure}}
\renewcommand{\thetable}{\thelecnum.\arabic{table}}

\newcommand{\aosv}{1044414: Advanced Operating Systems and Virtualization}
\newcommand{\wir}{1038137: Web Information Retrieval}
\newcommand{\va}{1052057: Visual Analytics}
\newcommand{\advprog}{1044416: Advanced Programming}
\newcommand{\dchpc}{1044399: Data Centers and High Perf. Computing}

\newcommand{\qu}[1]{\marginnote{\textcolor{cyan}{#1}}}


%
% The following macro is used to generate the header.
%
\newcommand{\lecture}[4]{
   \pagestyle{myheadings}
   \thispagestyle{plain}
   \newpage
   \setcounter{lecnum}{#4}
   \setcounter{page}{1}
   \noindent
   \begin{center}
   \framebox{
      \vbox{\vspace{2mm}
    \hbox to 7.4in { {\bf #1
    \hfill Spring 2018} }
       \vspace{4mm}
       \hbox to 7.4in { {\Large \hfill Lecture #4: #2  \hfill} }
       \vspace{2mm}
       \hbox to 7.4in { {\it Lecturer: #3 \hfill Scribe: Anxhelo Xhebraj} }
      \vspace{2mm}}
   }
   \end{center}
   \markboth{Lecture #4: #2}{Lecture #4: #2}

   \iffalse
   {\bf Note}: {\it LaTeX template courtesy of UC Berkeley EECS dept.}

   {\bf Disclaimer}: {\it These notes have not been subjected to the
   usual scrutiny reserved for formal publications.  They may be distributed
   outside this class only with the permission of the Instructor.}
   \vspace*{4mm}
   \fi
}
%
% Convention for citations is authors' initials followed by the year.
% For example, to cite a paper by Leighton and Maggs you would type
% \cite{LM89}, and to cite a paper by Strassen you would type \cite{S69}.
% (To avoid bibliography problems, for now we redefine the \cite command.)
% Also commands that create a suitable format for the reference list.
\iffalse
\renewcommand{\cite}[1]{[#1]}
\def\beginrefs{\begin{list}%
        {[\arabic{equation}]}{\usecounter{equation}
         \setlength{\leftmargin}{2.0truecm}\setlength{\labelsep}{0.4truecm}%
         \setlength{\labelwidth}{1.6truecm}}}
\def\endrefs{\end{list}}
\def\bibentry#1{\item[\hbox{[#1]}]}
\fi

%Use this command for a figure; it puts a figure in wherever you want it.
%usage: \fig{NUMBER}{SPACE-IN-INCHES}{CAPTION}
\newcommand{\fig}[3]{
            \vspace{#2}
            \begin{center}
            Figure \thelecnum.#1:~#3
            \end{center}
    }
% Use these for theorems, lemmas, proofs, etc.
\newtheorem{theorem}{Theorem}[lecnum]
\newtheorem{lemma}[theorem]{Lemma}
\newtheorem{proposition}[theorem]{Proposition}
\newtheorem{claim}[theorem]{Claim}
\newtheorem{corollary}[theorem]{Corollary}
\newtheorem{definition}[theorem]{Definition}
\newenvironment{proof}{{\bf Proof:}}{\hfill\rule{2mm}{2mm}}

% **** IF YOU WANT TO DEFINE ADDITIONAL MACROS FOR YOURSELF, PUT THEM HERE:

\newcommand\E{\mathbb{E}}

\begin{document}

\nocite{*}

%FILL IN THE RIGHT INFO.
%\lecture{**LECTURE-NUMBER**}{**DATE**}{**LECTURER**}{**SCRIBE**}

\lecture{\aosv}{March 27}{Alessandro Pellegrini}{8}

%\footnotetext{These notes are partially based on those of Nigel Mansell.}

% **** YOUR NOTES GO HERE:

Why is \texttt{\_\_syscall\_return} needed? A syscall can either fail or succeed. To let the user know what happened that macro sets \texttt{errno}. Signed arithmetic is more costly than unsigned therefore making the check unsigned the compiler will generate unsigned arithmetic instructions making the check faster.

Only 8 registers are available: ESP cannot be used, EAX is used for the code of the syscall, EBP must be saved before used. 

The dispatcher takes a complete snapshot of CPU registers for two reasons: the dispatcher will mangle the content of the registers so it has to save them; when the trap is issued the calling convention does not say anything about what registers could be modified by the kernel. 

The registers are pushed on stack and then a call to the syscall is performed. If the snapshot is modified then a different context will be given back to the user process. 

Resizing the syscall table (static variable vector) requires to reshuffle the whole compilation process since Bootmem must have a new bitmap etc.

\texttt{sysenter/sysexit} are fast system call paths since by using \texttt{int \$0x80} too much overhead was needed (enter IDT, GDT, change TR etc).

stdlib introduced \texttt{syscall()} interface but then later the instructions above were introduced in the kernel. 

Dispatcher

Model specific register speedup the entering of kernel mode since they store the informations needed to start executing kernel code. Doesn't do any check. 

Virtual Dynamic Shared Object: implements assembly code to activate/issue the 

\section{Multi cores synchronisation issues}

Changes in some data structures in the system we must ensure that all cores are aligned to same view of data structures and resources. Some issues are addressed by firmware, other not.

\marginnote{\textsc{Inter Processor Interrupts (IPI)}} 
Requests other cores to perform some action. Firmware generates them but software processes them. 

\newpage
\bibliography{Lec8}
\bibliographystyle{plainnat}
\end{document}