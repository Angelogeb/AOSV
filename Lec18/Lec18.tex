\documentclass[twoside]{article}
\setlength{\oddsidemargin}{-0.5 in}
\setlength{\evensidemargin}{1.5 in}
\setlength{\topmargin}{-0.6 in}
\setlength{\textwidth}{5.5 in}
\setlength{\textheight}{8.5 in}
\setlength{\headsep}{0.5 in}
\setlength{\parindent}{0 in}
\setlength{\parskip}{0.07 in}
\setlength{\marginparwidth}{145pt}

%
% ADD PACKAGES here:
% 12

\usepackage{amsmath,
            amsfonts,
            amssymb,
            graphicx,
            mathtools,
            flexisym,
            marginnote,
            hyperref,
            titlesec}

\usepackage[english]{babel}
\usepackage[utf8]{inputenc}
\usepackage[shortlabels]{enumitem}

\graphicspath{ {images/} }

\hypersetup{
    colorlinks=true,
    linkcolor=blue,
    filecolor=magenta,      
    urlcolor=blue,
}

\titlespacing\section{0pt}{12pt plus 4pt minus 2pt}{0pt plus 2pt minus 2pt}
\titlespacing\subsection{0pt}{12pt plus 4pt minus 2pt}{0pt plus 2pt minus 2pt}

%
% The following commands set up the lecnum (lecture number)
% counter and make various numbering schemes work relative
% to the lecture number.
%
\newcounter{lecnum}
\renewcommand{\thepage}{\thelecnum-\arabic{page}}
\renewcommand{\thesection}{\thelecnum.\arabic{section}}
\renewcommand{\theequation}{\thelecnum.\arabic{equation}}
\renewcommand{\thefigure}{\thelecnum.\arabic{figure}}
\renewcommand{\thetable}{\thelecnum.\arabic{table}}

\newcommand{\aosv}{1044414: Advanced Operating Systems and Virtualization}
\newcommand{\wir}{1038137: Web Information Retrieval}
\newcommand{\va}{1052057: Visual Analytics}
\newcommand{\advprog}{1044416: Advanced Programming}
\newcommand{\dchpc}{1044399: Data Centers and High Perf. Computing}

\newcommand{\qu}[1]{\marginnote{\textcolor{cyan}{#1}}}


%
% The following macro is used to generate the header.
%
\newcommand{\lecture}[4]{
   \pagestyle{myheadings}
   \thispagestyle{plain}
   \newpage
   \setcounter{lecnum}{#4}
   \setcounter{page}{1}
   \noindent
   \begin{center}
   \framebox{
      \vbox{\vspace{2mm}
    \hbox to 7.4in { {\bf #1
    \hfill Spring 2018} }
       \vspace{4mm}
       \hbox to 7.4in { {\Large \hfill Lecture #4: #2  \hfill} }
       \vspace{2mm}
       \hbox to 7.4in { {\it Lecturer: #3 \hfill Scribe: Anxhelo Xhebraj} }
      \vspace{2mm}}
   }
   \end{center}
   \markboth{Lecture #4: #2}{Lecture #4: #2}

   \iffalse
   {\bf Note}: {\it LaTeX template courtesy of UC Berkeley EECS dept.}

   {\bf Disclaimer}: {\it These notes have not been subjected to the
   usual scrutiny reserved for formal publications.  They may be distributed
   outside this class only with the permission of the Instructor.}
   \vspace*{4mm}
   \fi
}
%
% Convention for citations is authors' initials followed by the year.
% For example, to cite a paper by Leighton and Maggs you would type
% \cite{LM89}, and to cite a paper by Strassen you would type \cite{S69}.
% (To avoid bibliography problems, for now we redefine the \cite command.)
% Also commands that create a suitable format for the reference list.
\iffalse
\renewcommand{\cite}[1]{[#1]}
\def\beginrefs{\begin{list}%
        {[\arabic{equation}]}{\usecounter{equation}
         \setlength{\leftmargin}{2.0truecm}\setlength{\labelsep}{0.4truecm}%
         \setlength{\labelwidth}{1.6truecm}}}
\def\endrefs{\end{list}}
\def\bibentry#1{\item[\hbox{[#1]}]}
\fi

%Use this command for a figure; it puts a figure in wherever you want it.
%usage: \fig{NUMBER}{SPACE-IN-INCHES}{CAPTION}
\newcommand{\fig}[3]{
            \vspace{#2}
            \begin{center}
            Figure \thelecnum.#1:~#3
            \end{center}
    }
% Use these for theorems, lemmas, proofs, etc.
\newtheorem{theorem}{Theorem}[lecnum]
\newtheorem{lemma}[theorem]{Lemma}
\newtheorem{proposition}[theorem]{Proposition}
\newtheorem{claim}[theorem]{Claim}
\newtheorem{corollary}[theorem]{Corollary}
\newtheorem{definition}[theorem]{Definition}
\newenvironment{proof}{{\bf Proof:}}{\hfill\rule{2mm}{2mm}}

% **** IF YOU WANT TO DEFINE ADDITIONAL MACROS FOR YOURSELF, PUT THEM HERE:

\newcommand\E{\mathbb{E}}


\begin{document}

\nocite{*}

%FILL IN THE RIGHT INFO.
%\lecture{**LECTURE-NUMBER**}{**DATE**}{**LECTURER**}{**SCRIBE**}

\lecture{\aosv}{May 11}{Alessandro Pellegrini}{18}

%\footnotetext{These notes are partially based on those of Nigel Mansell.}

% **** YOUR NOTES GO HERE:

\section{Module Parameters}
\label{sec:Module Parameters}

Parameters might define how the module should behave when executing. In linux,
to have variables passed to modules you define global variables and "register"
them.

To define global variables of a certain type as module parameters we can use
\texttt{module_param}.

When declaring these parameters some objects/pseudofiles are placed in Sysfs. We
have to define the permission mask.


Internal modules are the ones belonging to the main tree of the linux kernel and
are included by default in \texttt{/lib/modules}. While external kernel modules
are external piece of codes. \texttt{insmod} allows to insert the external
module while \texttt{modprobe} checks \texttt{/lib/modules}. \texttt{rmmod}
unloads the module.


\section{Kernel Module Loading}
\label{sec:Kernel Module Loading}

The kernel allocates some memory to load both code and data structures. This
piece of code is subject to relocation. The memory is allocated through
\texttt{vmalloc} therefore it might not be contiguous in RAM but just virtually.

In kernel 2.4 the relocation was much more limited. So most of the job of the
relocation was done in userspace. When applications wanted to load kernel
modules they first resolved symbols and then loaded it in ram.

From kernel 2.6 the resolution of symbols is done by the kernel. The set of
tools shown previously trigger the kernel actions for memory allocation, module
loading and address resolution.

Kbuild for building the kernel modules. A \texttt{.ko} file starts with a normal
object file. \texttt{modpost} creates an additional file with extension
\texttt{.mod} that has information about the symbols of the \texttt{.o} file.


\texttt{EXPORT_SYMBOL} tells to Kbuild that it should take into account to make
"visible" some symbols of the kernel for module developers to interact with the
kernel.

\texttt{kprobes} allow to inject code into the kernel for monitoring performance
of kernel code. With kprobe everything will be probed except what is marked as
not to be. Through \texttt{INT 3} breakpoints are introduced. It requires only
one byte differently from other interrupts. Its code is \texttt{0xcc}.


Setting \texttt{TF} flag the processor will stop execution at each instruction.
\texttt{pushf/popf} allows to set the TF flag.

Prob handlers are run with preemption disabled.


\texttt{flush_tlb_all} is not exported. To find its virtual address we can use
kprob to get it.


\section{Kernel Messaging}
\label{sec:Kernel Messaging}


Specific modules that expose facilities to print on screen or files.

\texttt{printk} similar in spirit to printf. Accepts a format string and the
variables needed to be printed. \texttt{printk("\%pk", .. )} allows to print the real
virutal address of some kernel memory. When issuing a call to \texttt{printk} we
have to specify a priority.

The different loglevels tell what the kernel has to do with messages that were
generated or messages with no priority level specified.


\texttt{/proc/sys/kernel/printk} keeps 4 integers about the various log levels.
current, defualt, min etc.

The prints are managed through a ring buffer.



\texttt{automake}



\newpage
\bibliography{Lec18}
\bibliographystyle{plainnat}
\end{document}
